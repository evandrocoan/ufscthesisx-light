%https://github.com/SublimeText/LaTeXTools/issues/1439
%!TEX output_directory=latexcache

% ------------------------------------------------------------------------
% ------------------------------------------------------------------------
% Modelo UFSC para Trabalhos Academicos (tese de doutorado, dissertação de
% mestrado) utilizando a classe abntex2
%
% Autor: Alisson Lopes Furlani
% 	Modificações:
%	- 27/08/2019: Alisson L. Furlani, add pacote 'glossaries' para listas
% ------------------------------------------------------------------------
% ------------------------------------------------------------------------

\documentclass[
	% -- opções da classe memoir --
	12pt,				% tamanho da fonte
	%openright,			% capítulos começam em pág ímpar (insere página vazia caso preciso)
	oneside,			% para impressão no anverso. Oposto a twoside
	a4paper,			% tamanho do papel.
	% -- opções da classe abntex2 --
	chapter=TITLE,		% títulos de capítulos convertidos em letras maiúsculas
	section=TITLE,		% títulos de seções convertidos em letras maiúsculas
	%subsection=TITLE,	% títulos de subseções convertidos em letras maiúsculas
	%subsubsection=TITLE,% títulos de subsubseções convertidos em letras maiúsculas
	% -- opções do pacote babel --
	english,			% idioma adicional para hifenização
	%french,				% idioma adicional para hifenização
	%spanish,			% idioma adicional para hifenização
	brazil,
	brazilian, % o último idioma é o principal do documento
	]{setup/ufscthesisx}

\addbibresource{aftertext/references.bib} % Seus arquivos de referências

% ---
% Filtering and Mapping Bibliographies
% ---
\DeclareSourcemap{
	\maps[datatype=bibtex]{
		% remove fields that are always useless
		\map{
			\step[fieldset=abstract, null]
			\step[fieldset=pagetotal, null]
		}
		% remove URLs for types that are primarily printed
%		\map{
%			\pernottype{software}
%			\pernottype{online}
%			\pernottype{report}
%			\pernottype{techreport}
%			\pernottype{standard}
%			\pernottype{manual}
%			\pernottype{misc}
%			\step[fieldset=url, null]
%			\step[fieldset=urldate, null]
%		}
		\map{
			\pertype{inproceedings}
			% remove mostly redundant conference information
			\step[fieldset=venue, null]
			\step[fieldset=eventdate, null]
			\step[fieldset=eventtitle, null]
			% do not show ISBN for proceedings
			\step[fieldset=isbn, null]
			% Citavi bug
			\step[fieldset=volume, null]
		}
	}
}
% ---

% ---
% Informações de dados para CAPA e FOLHA DE ROSTO
% ---
% FIXME Substituir 'Nome completo do autor' pelo seu nome.
\autor{Nome completo do autor(a)}
% FIXME Substituir 'Título do trabalho' pelo título da trabalho.
\titulo{Título do trabalho}
% FIXME Substituir 'Subtítulo (se houver)' pelo subtítulo da trabalho.
% Caso não tenha substítulo, comente a linha a seguir.
\subtitulo{Subtítulo (se houver)}
% FIXME Substituir 'XXXXXX' pelo nome do seu
% orientador.
\orientador{Prof. XXXXXX, Dr.}
% FIXME Se for orientado por uma mulher, comente a linha acima e descomente a linha a seguir.
% \orientador[Orientadora]{Nome da orientadora, Dra.}
% FIXME Substituir 'XXXXXX' pelo nome do seu
% coorientador. Caso não tenha coorientador, comente a linha a seguir.
\coorientador{Prof. XXXXXX, Dr.}
% FIXME Se for coorientado por uma mulher, comente a linha acima e descomente a linha a seguir.
% \coorientador[Coorientadora]{XXXXXX, Dra.}
% FIXME Substituir 'XXXXXX' pelo nome do Coordenador do
% programa/curso.
\coordenador{Prof. XXXXXX, Dr.}
% FIXME Se for coordenadora mulher, comente a linha acima e descomente a linha a seguir.
% \coordenador[Coordenadora]{Nome da Coordenadora, Dra.}
% FIXME Substituir '[ano]' pelo ano (ano) em que seu trabalho foi defendido.
\ano{ano}
% FIXME Substituir '[dia] de [mês] de [ano]' pela data em que ocorreu sua defesa.
\data{[dia] de [mês] de [ano]}
% FIXME Substituir 'Local' pela cidade em que ocorreu sua defesa.
\local{Local}
\instituicaosigla{UFSC}
\instituicao{Universidade Federal de Santa Catarina}
% FIXME Substituir 'Dissertação/Tese' pelo tipo de trabalho (Tese, Dissertação).
\tipotrabalho{Dissertação/Tese}
% FIXME Substituir '[mestre/doutor] em XXXXXX' pela grau adequado.
\formacao{[mestre/doutor] em [nome do título obtido pelo Programa]}
% FIXME Substituir '[mestrado/doutorado]' pelo nivel adequado.
\nivel{[mestrado/doutorado]}
% FIXME Substituir 'Programa de Pós-Graduação em XXXXXX' pela curso adequado.
\programa{Programa de Pós-Graduação em XXXXXX}
% FIXME Substituir 'Campus XXXXXX ou Centro de XXXXXX' pelo campus ou centro adequado.
\centro{Campus XXXXXX ou Centro de XXXXXX}
% FIXME: Preencha com Campus XXXXXX     ou     Centro de XXXXXX
\campus{Campus Reitor João David Ferreira Lima}
\preambulo
{%
\imprimirtipotrabalho~submetida~ao~\imprimirprograma~da~\imprimirinstituicao~para~a~obtenção~do~título~de~\imprimirformacao.
}
% ---

% ---
% Configurações de aparência do PDF final
% ---
% alterando o aspecto da cor azul
\definecolor{blue}{RGB}{41,5,195}
% informações do PDF
\makeatletter
\hypersetup{
     	%pagebackref=true,
		pdftitle={\@title},
		pdfauthor={\@author},
    	pdfsubject={\imprimirpreambulo},
	    pdfcreator={LaTeX with abnTeX2},
		pdfkeywords={ufsc, latex, abntex2},
		colorlinks=true,       		% false: boxed links; true: colored links
    	linkcolor=black,%blue,          	% color of internal links
    	citecolor=black,%blue,        		% color of links to bibliography
    	filecolor=black,%magenta,      		% color of file links
		urlcolor=black,%blue,
		bookmarksdepth=4
}
\makeatother
% ---

% ---
% compila a lista de abreviaturas e siglas e a lista de símbolos
% ---

% Declaração das siglas
\siglalista{ABNT}{Associação Brasileira de Normas Técnicas}

% Declaração dos simbolos
\simbololista{C}{\ensuremath{C}}{Circunferência de um círculo}
\simbololista{pi}{\ensuremath{\pi}}{Número pi}
\simbololista{r}{\ensuremath{r}}{Raio de um círculo}
\simbololista{A}{\ensuremath{A}}{Área de um círculo}

% compila a lista de abreviaturas e siglas e a lista de símbolos
\makenoidxglossaries

% ---

% ---
% compila o indice
% ---
\makeindex
% ---

% ----
% Início do documento
% ----
\begin{document}

% Seleciona o idioma do documento (conforme pacotes do babel)
%\selectlanguage{english}
\selectlanguage{brazil}

% Retira espaço extra obsoleto entre as frases.
\frenchspacing

% Espaçamento 1.5 entre linhas
\OnehalfSpacing

% Corrige justificação
%\sloppy

% ----------------------------------------------------------
% ELEMENTOS PRÉ-TEXTUAIS
% ----------------------------------------------------------
% \pretextual %a macro \pretextual é acionado automaticamente no início de \begin{document}
% ---
% Capa, folha de rosto, ficha bibliografica, errata, folha de apróvação
% Dedicatória, agradecimentos, epígrafe, resumos, listas
% ---
\input{beforetext/beforetext}
% ---

% ----------------------------------------------------------
% ELEMENTOS TEXTUAIS
% ----------------------------------------------------------
\textual

% ---
% 1 - Introdução
% ---
\include{chapters/1-chapter}
% ---

% ---
% 2 - Capítulo 2
% ---
\include{chapters/2-chapter}
% ---

% ---
% 3 - Capítulo 3
% ---
\include{chapters/3-chapter}
% ---

% ---
% 4 - Conclusão
% ---
%\phantompart
\include{chapters/4-chapter}
% ---

% ----------------------------------------------------------
% ELEMENTOS PÓS-TEXTUAIS
% ----------------------------------------------------------
%\postextual
% ----------------------------------------------------------

% ----------------------------------------------------------
% Referências bibliográficas
% ----------------------------------------------------------
\begingroup
    \SingleSpacing
	\setlength\bibitemsep{\baselineskip}
    \printbibliography[title=REFERÊNCIAS]
\endgroup

% ----------------------------------------------------------
% Glossário
% ----------------------------------------------------------
%
% Consulte o manual da classe abntex2 para orientações sobre o glossário.
%
%\glossary

% ----------------------------------------------------------
% Apêndices
% ----------------------------------------------------------

% ---
% Inicia os apêndices
% ---
\begin{apendicesenv}
%	\partapendices*
	\input{aftertext/apendice_a}
\end{apendicesenv}
% ---


% ----------------------------------------------------------
% Anexos
% ----------------------------------------------------------

% ---
% Inicia os anexos
% ---
\begin{anexosenv}
%	\partanexos*
	\input{aftertext/anexo_a}
\end{anexosenv}

%---------------------------------------------------------------------
% INDICE REMISSIVO
%---------------------------------------------------------------------
%\phantompart
%\printindex
%---------------------------------------------------------------------

\end{document}
